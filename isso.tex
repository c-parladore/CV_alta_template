%%%%%%%%%%%%%%%%%
% This is an example CV created using altacv.cls (v1.1.5, 1 December 2018) written by 
% LianTze Lim (liantze@gmail.com), based on the
% Cv created by BusinessInsider at http://www.businessinsider.my/a-sample-resume-for-marissa-mayer-2016-7/?r=US&IR=T
%
%% It may be distributed and/or modified under the
%% conditions of the LaTeX Project Public License, either version 1.3
%% of this license or (at your option) any later version.
%% The latest version of this license is in
%%    http://www.latex-project.org/lppl.txt
%% and version 1.3 or later is part of all distributions of LaTeX
%% version 2003/12/01 or later.
%%%%%%%%%%%%%%%%

%% If you are using \orcid or academicons
%% icons, make sure you have the academicons
%% option here, and compile with XeLaTeX
%% or LuaLaTeX.
% \documentclass[10pt,a4paper,academicons]{altacv}

%% Use the "normalphoto" option if you want a normal photo instead of cropped to a circle
% \documentclass[10pt,a4paper,normalphoto]{altacv}

\documentclass[10pt,a4paper,ragged2e,academicons]{altacv}


\geometry{left=1cm,right=9cm,marginparwidth=6.8cm,marginparsep=1.2cm,top=1.25cm,bottom=1.25cm}

% Change the font if you want to, depending on whether
% you're using pdflatex or xelatex/lualatex
\ifxetexorluatex
  % If using xelatex or lualatex:
  \setmainfont{Carlito}
\else
  % If using pdflatex:
  \usepackage[utf8]{inputenc}
  \usepackage[T1]{fontenc}
  \usepackage[default]{lato}
\fi

\usepackage{hyperref}

% Change the colours if you want to
\definecolor{VividPurple}{HTML}{003E97}
\definecolor{SlateGrey}{HTML}{2E2E2E}
\definecolor{LightGrey}{HTML}{666666}
\colorlet{heading}{VividPurple}
\colorlet{accent}{VividPurple}
\colorlet{emphasis}{SlateGrey}
\colorlet{body}{LightGrey}

% Change the bullets for itemize and rating marker
% for \cvskill if you want to
\renewcommand{\itemmarker}{{\small\textbullet}}
\renewcommand{\ratingmarker}{\faCircle}

%% sample.bib contains your publications
\addbibresource{sample.bib}
%-DOCUMENT-----------------------------------------------------
\begin{document}
\name{Cezar Parladore}
\tagline{}
\photo{3cm}{me-2}
\personalinfo{%
  % Not all of these are required!
  % You can add your own with \printinfo{symbol}{detail}
  \location{Rua Helvetia, São Paulo, SP}
  \email{cezarparladore@outlook.com}
    \phone{(11) 94401-0004}
   \linkedin{linkedin.com/in/cezar-parladore}
    \github{github.com/c-parladore}
    % I'm just making this up though.
%   \orcid{orcid.org/0000-0000-0000-0000} % Obviously making this up too. If you want to use this field (and also other academicons symbols), add "academicons" option to \documentclass{altacv}
}

%% Make the header extend all the way to the right, if you want.
\begin{fullwidth}
\makecvheader
\end{fullwidth}

%% Depending on your tastes, you may want to make fonts of itemize environments slightly smaller
\AtBeginEnvironment{itemize}{\small}

%% Provide the file name containing the sidebar contents as an optional parameter to \cvsection.
%% You can always just use \marginpar{...} if you do
%% not need to align the top of the contents to any
%% \cvsection title in the "main" bar.

%-EXPERIENCIA PROFISSIONAL-------------------------
\cvsection[page1sidebar]{Experiência Profissional}

\cvevent{Assistente Administrativo}{Worley/Advisian}{Jan, 2021-atualmente}{São Paulo}
\begin{itemize}
    \item Manipulação, organização e transformação de dados utilizando \textbf{Excel, Power Query, RStudio e Tidyverse}.
    \item Criação de \textit{dashboards} em \textbf{PowerBI} (desktop e para publicação no PowerBI Service) para visualização de dados ambientais (e.g. resultados analíticos e água e solo, dados de meio físico e descrições de sondagens).
    \item Tratamento de dados laboratoriais para \textit{input} em banco de dados do\textbf{EQuIS}\faRegistered.
    \item Elaboração de relatórios de monitoramento de água subterrânea, avaliação de risco à saúde humana, investigação preliminar e testes pilotos de remediação.
    \item Participou de um projeto piloto de \textbf{fitorremediação} atuando na o\textbf{organização, tratamento e visualização dos dados gerados em 18 campanhas} de amostragem de água e solo.
    \item Desenho vetorial de perfis de sondagem, seções geológicas e plumas de contaminação no \textbf{Inkscape}.

\end{itemize}
\divider


\cvevent{Estagiário de Meio Ambiente}{Jacobs - Worley}{Fev, 2019 - Dez, 2020}{São Paulo}
\begin{itemize}
    \item Realizou \textbf{6 meses de rotatividad}e pelos setores da empresa participando de projetos em saneamento, abastecimento de água, licenciamento ambiental e gerenciamento de áreas contaminadas.
    \item Auxílio na elaboração de \textbf{relatórios de monitoramento} de água subterrânea, tratamento de dados laboratoriais e de campo e revisão de dados históricos. 
    \item Vetorização de imagens em GIS.
    \item Confecção de \textbf{seções geológicas, mapas potenciométricos, plumas de contaminantes} e perfis de sondagem no \textbf{Inkscape}.
    \item Revisão bibliográfica sobre geologia regional e hidrogeologia para capítulos de relatórios.
\end{itemize}
\divider

\begin{fullwidth}

\cvevent{Diretor de RH; 2° Secretário}{GeoJunior Consultoria}{2016-1017}{IGc-USP}
\begin{itemize}
    \item Liderou a organização de um processo seletivo de novos membros e participou ativamente na reestruturação administrativa e na renovação da identidade visual da empresa.
    
\end{itemize}


%-PROJETOS---------------------------------
\cvsection{Projetos }

\cveventnewf{Avaliação do Uso de Pó de Rocha para Remineralização de Solo no Assentamento Ipanema, Iperó, SP}{Trabalho de Conclusão de Curso}{2021-em andamento}
\begin{itemize}
    \item Developed novel \emph{masked punctuation prediction} technique coupled with addition of \texttt{[PAUSE]} tokens,\\achieving state-of-the-art on Multi-Genre Broadcast (MGB) dataset and promising results on the IWSLT 2011 dataset.
    \item Aiming to publish subset of findings in \emph{ICASS 2022}.
\end{itemize}


\end{fullwidth}

% \cvsection[page2sidebar]{Publications}
% 
% \nocite{*}
% 
% \printbibliography[heading=pubtype,title={\printinfo{\faBook}{Books}},type=book]
% 
% \divider
% 
% \printbibliography[heading=pubtype,title={\printinfo{\faFileTextO}{Journal Articles}}, type=article]
% 
% \divider
% 
% \printbibliography[heading=pubtype,title={\printinfo{\faGroup}{Conference Proceedings}},type=inproceedings]
% 
%% If the NEXT page doesn't start with a \cvsection but you'd
%% still like to add a sidebar, then use this command on THIS
%% page to add it. The optional argument lets you pull up the
%% sidebar a bit so that it looks aligned with the top of the
%% main column.
% \addnextpagesidebar[-1ex]{page3sidebar}


\end{document}
